\documentclass[a4paper,12pt]{article}
\usepackage[utf8]{inputenc}
\usepackage[francais]{babel}
\usepackage[margin=2cm]{geometry}
\usepackage[T1]{fontenc}

\title{ELIZA}
\author{Lucas Erard ~~\\ Marie Lavigne~~\\ Thibaud Levasseur}

\begin{document}
\maketitle
\newpage
\tableofcontents
\newpage
\section{Introduction}
Ce projet a pour but de concevoir et développer un "ChatBOT", ou robot de discussion textuel, qui doit permettre à un utilisateur de discuter avec l'ordinateur via le clavier. Le modèle mis à notre disposition à l'adresse http://eliza.levillage.org/ est une application simulant une conversation avec un psychologue.
Le programme ELIZA est un programme développé en 1966 par Joseph Weizenbaum (1923-2008) pour simuler une interaction avec un psychothérapeute rogérien. Le but de ce programme est donc de reformuler les phrases du patient en question.
Ce programme à la base écrit pour des anglophones a été transposé vers le français avec un succès beaucoup plus que mitigé, dû à une syntaxe plus libre.
Le but de ce programme n'est en fait pas de répondre à proprement parler avec un interlocuteur mais de le faire parler, lui donner l'impression d'être écouté.
C'est donc sur ce modèle que nous devrons réaliser notre projet SOSI.
\section{Conception}
\section{Développement}
\section{Annexes}
\end{document}
