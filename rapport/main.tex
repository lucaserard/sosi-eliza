\documentclass[a4paper,12pt]{article}
\usepackage[utf8]{inputenc}
\usepackage[francais]{babel}
\usepackage[margin=2cm]{geometry}
\usepackage[T1]{fontenc}
\usepackage{graphicx}
\title{ELIZA}
\author{Lucas Erard ~~\\ Marie Lavigne~~\\ Thibaud Levasseur}

\begin{document}
\maketitle
\newpage
\tableofcontents
\newpage
\section{Introduction}
Ce projet a pour but de concevoir et développer un "ChatBOT", ou robot de discussion textuel, qui doit permettre à un utilisateur de discuter avec l'ordinateur via le clavier. Le modèle mis à notre disposition à l'adresse http://eliza.levillage.org/ est une application simulant une conversation avec un psychologue.
Le programme ELIZA est un programme développé en 1966 par Joseph Weizenbaum (1923-2008) pour simuler une interaction avec un psychothérapeute rogérien. Le but de ce programme est donc de reformuler les phrases du patient en question.
Ce programme à la base écrit pour des anglophones a été transposé vers le français avec un succès  plus que mitigé, dû à une syntaxe plus libre.
Le but de ce programme n'est en fait pas de répondre à proprement parler avec un interlocuteur mais de le faire parler, lui donner l'impression d'être écouté.
C'est donc sur ce modèle que nous devrons réaliser notre projet SOSI.
\section{Conception}
Notre premier choix de conception était de réaliser ce projet à l'aide des interpréteurs de commandes Lex et Yacc sur lesquels nous avions travaillé lors du cours d'introduction à la compilation du semestre dernier. Cependant, il nous est rapidement apparut que ce choix, bien que probablement le meilleur, risquait de demander trop de temps. Par conséquent, afin de respecter nos délais tout en rendant tout de même le meilleur projet possible, nous nous sommes redirigés vers le Java, un langage que nous avons aussi pratiqué le semestre précédent. 

Une fois le langage choisi, nous nous sommes mis à la répartition des tâches. Ainsi, tandis que Lucas et Thibaud se sont penchés sur l'analyse des phrase et la réponse, Marie a codé tout ce qui était en rapport avec l'IHM (récupération des phrases envoyées, affichage des réponses et interface graphique). 
\section{Développement}
\subsection{Idée Originale : Bison} Nous avons tout d'abord pensé à developper notre agent conversationnel sous la forme d'un interpréteur, en utilisant les technologies Lex et Yacc vues dans le cours d'Intro à la Compilation. Cet interpréteur aurait analysé la phrase entrée par l'utilisateur à l'aide des règles de la grammaire française, puis construit une réponse appropriée à l'aide de ces mêmes règles.Une telle analyse aurait nécessité de distinguer tous les agents grammaticaux d'une phrase (sujet, verbe, COD/COI ...), tout en repérant les incises et subordonnées. Bien évidemment, face à  l'ampleur de la tache, le peu de temps dont nous disposions et notre maitrise limitée de la technologie nous ont amenés à envisager une autre solution.

\subsection{Mise en pratique : Java}
Nous avons donc suivi les conseils de notre enseignant référent Alexande Pauchet, et nous sommes tournés vers une technologie que nous connaissions bien mieux : le Java.
\paragraph{}La logique métier a été codée de manière très simple, selon les principes vus en cours d'Algorithmique et de Programmation Avancée : une classe pour chaque Type Abstrait de Données, et une pour l'Intelligence Artificielle.
\paragraph{}Concernant l'interface, la première phase du développement a consisté à simplement faire une classe pour récupérer le texte entré en console puis afficher une réponse. En effet, nous souhaitions être sûrs que, même si l'interface n'était pas réalisée dans les temps ou ne fonctionnait pas bien, nous aurions tout de même la possibilité d'afficher le dialogue via la console. 
Pour ce dialogue console, nous avons d'abord voulu utiliser un Scanner, comme nous l'avions fait dans notre projet de Programmation Avancée au semestre 6 pour ensuite y préférer un BufferedReader qui nous a été conseillé et, après test, s'est avéré mieux fonctionner.

Nous avons ensuite pu passer à la réalisation de l'interface en elle-même, dont Marie s'est chargée. 
\paragraph{Marie : }Ainsi, j'ai souhaité utiliser eclipse et son outil de création d'interface, WindowBuilder. Comme ce logiciel nous était interdit l'année précédente, j'ai dû réapprendre à l'utiliser et j'ai aussi totalement découvert son plugin WindowBuilder. Afin de prendre en main ce "constructeur", j'ai passé du temps à chercher de la documentation à son sujet. Il est apparut que bien qu'étant un outil très pratique pour la visualisation et la création basique de fenêtres, il ne nous absous pas de devoir coder puisqu'en général, nous souhaitons personnaliser bien des aspects de celui-ci selon ce que nous devons réaliser. J'ai donc jonglé entre cet outil et le code afin d'obtenir une interface correspondant au mieux aux besoins de notre projet.
 
\section{Annexes}
\end{document}
